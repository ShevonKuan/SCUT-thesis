%%
% 引言或背景
% 引言是论文正文的开端,应包括毕业论文选题的背景、目的和意义;对国内外研究现状和相关领域中已有的研究成果的简要评述;介绍本项研究工作研究设想、研究方法或实验设计、理论依据或实验基础;涉及范围和预期结果等。要求言简意赅,注意不要与摘要雷同或成为摘要的注解。
%%

\chapter{绪论}
%定义,过去的研究和现在的研究,意义,与图像分割的不同,going deeper
\label{cha:introduction}
\section{引言}
\label{sec:background}
当今社会,科技的飞速发展为大家提供了快捷与舒适,但与此同时也增添了在信息安全上的危险。在过去的二十几年来,我们通过数字密码来鉴别身份,但是随着科技的发展,不法分子借用高科技犯罪的案例年年增高,密码被盗的情况时常发生。因此,怎样科学准确的辨别每一个人的身份则成为当今社会的重要问题。
\section{研究背景}
\label{sec:related_work}
随着科技的日益发展,传统的密码因为记忆的繁琐以及容易被盗,似乎已经不再能满足这个通信发达的社会的需求。人们急需一种更便捷而且辨识度更高的方式来辨识身份。循着便捷与辨识度高这两个约束条件\overcite{ref1},我们联想到的便是存在于每个人身上的生物特征,所以基于每个人身上不同的生物特征而研究的鉴别技术现在成为了身份辨别技术上的主流。

\section{研究现状}
笔迹获取的方式有两种,所以鉴别方式也分为离线鉴别和在线鉴别\overcite{ref2,ref3}。在线鉴别是采用专用的数字板来实时收集书写信号。由文献\cite{ref4,ref5,ref6,ref7}可知,因为信号是实时采集的,所以能采集的数据不仅包括笔迹序列,而且可以采集到书写时的加速度、压力、速度等丰富有用的动态信息。

\section{论文结构}
本文分为四章。其中第一章简述了笔迹识别的研究背景和意义以及笔迹识别的基础知识等。第二章节从卷积神经网络的发展历史、网络结构、学习规律三方面详细的讲述了卷积网络的基础知识。第三章针对本文中的手写数字及写字人实验具体设计卷积神经网络的网络结构以及训练过程。第五章节是手写数字识别及写字人识别实验的结果与分析。